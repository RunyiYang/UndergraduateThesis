\BITSetup{
  cover = {
    % 在封面中载入有「北京理工大学」字样的图片,如无必要请勿改动。
    headerImage = images/header.png,
    % 在封面标题中使用思源黑体,使用此选项可以保证与 Word 封面标题的字体一致。
    xiheiFont = STXIHEI.TTF,
    % 官方模板采用了固定的下划线宽度。我们采用以下两个选项来达成这个效果。
    % 如果你想要使用自动计算的下划线宽度,也可以删去以下两个选项。
    autoWidth = false,
    valueMaxWidth = 20em,
  },
  info = {
    title = 基于深度学习的端到端多实例点云配准,
    titleEn = {Deep Learning Based End-To-End \\ Multi-instance Point Cloud Registration},
    % 想要删除某项封面信息,直接删除该项即可。
    % 想要让某项封面信息留空(但是保留下划线),请设置为空字符串。
    % 如需要换行,则用 “\\” 符号分割。
    school = 自动化学院,
    major = 自动化,
    author = 杨润一,
    class = 06011902,
    studentId = 1120191211,
    supervisor = 由育阳,
    translationTitle = 多实例点云配准的高效对应聚类方法,
    translationOriginTitle = Multi-instance Point Cloud Registration \\ by Efficient Correspondence Clustering,
    keywords = {点云配准; 多实例; 聚类; 对应聚类; 深度学习;毕业设计(外文翻译)},
    % 如果你的毕设为校外毕设,请将下面这一行语句解除注释(删除第一个百分号字符)并填写你的校外毕设导师名字
  },
  style = {
    % 保持参考文献的缩进样式与 Word 模板一致。
    % 如果你不需要此样式,请将此行注释掉。
    bibliographyIndent = false,
    % head = {自定义页眉文字}
  }
}

% 有关参考文献的样式可以在此处修改;如无必要无需修改。
\usepackage[
  backend=biber,
  style=gb7714-2015,
  gbalign=gb7714-2015,
  gbnamefmt=lowercase,
  gbpub=false,
  doi=false,
  url=false,
  eprint=false,
  isbn=false,
]{biblatex}

\usepackage{caption}
\usepackage{subcaption}
\usepackage{graphicx}
\usepackage{amsmath}
\usepackage{amssymb}
\usepackage{amsfonts}
\usepackage{booktabs}
\usepackage{gensymb}