%%
% The BIThesis Template for Bachelor Paper Translation
%
% 北京理工大学毕业设计(论文) —— 使用 XeLaTeX 编译
%
% Copyright 2020-2023 BITNP
%
% This work may be distributed and/or modified under the
% conditions of the LaTeX Project Public License, either version 1.3
% of this license or (at your option) any later version.
% The latest version of this license is in
%   http://www.latex-project.org/lppl.txt
% and version 1.3 or later is part of all distributions of LaTeX
% version 2005/12/01 or later.
%
% This work has the LPPL maintenance status `maintained'.
%
% The Current Maintainer of this work is Feng Kaiyu.
%
% Compile with: xelatex -> biber -> xelatex -> xelatex
%%

% 第一章节

\chapter{相关工作}

\textbf{点云配准}可以分为三个阶段:点匹配、异常值剔除和姿态估计。大多数工作集中在前两个阶段,因为获得正确的点对应关系是成功配准的关键。点匹配通常依赖于特征,可以是手工设计的特征\cite{rusu2009fast, drost2010model},也可以是基于学习的特征\cite{3DMatch, PPFNet, 3DSmoothNet, FCGF, D3Feat}。尽管近期的结果显示,在某些基准测试中,后者优于手工设计的特征,但这些特征仍然无法实现完美匹配,因此仍然需要强大的异常值剔除机制。
RANSAC\cite{RANSAC}及其变体(\cite{GCRansac, NGRansac, LORansac})遵循假设-验证过程来剔除异常值。当存在许多异常值时,这类方法需要大量的采样步骤,耗时严重,而且仍可能无法获得正确的模型。GORE\cite{bustos2017guaranteed}和PMC\cite{parra2019practical}试图通过几何一致性检查来减少异常值。其他方法如FGR\cite{FGR}和TEASER\cite{TEASER}采用鲁棒估计器直接从噪声对应关系中解决变换问题。通过仔细处理每个子问题,TEASER\cite{TEASER}在鲁棒性和效率方面取得了令人印象深刻的性能。
还有一些基于学习的异常值剔除方法。DGR\cite{DGR}和3DRegNet\cite{3DRegNet}将异常值剔除视为二分类问题,并预测每个对应关系的内点概率。PointDSC\cite{PointDSC}进一步将空间一致性嵌入到特征学习中,以便更好地训练内点分类器。
最近,一系列工作(如PointNetLK\cite{PointNetLK},FMR\cite{FMR},DCP\cite{DCP},PRNet\cite{PRNet},RPMNet\cite{RPMNet})尝试将端到端学习应用于解决配准问题。它们在性能方面也表现出色,特别是在低重叠情况下\cite{PREDATOR}。

现有的点云配准方法主要关注一对一配准问题,即估计两个点云之间的单一变换。然而,将源点云与目标点云中的多个实例对齐的多实例配准问题则研究较少。这个任务与多路配准\cite{choi2015robust}不同,后者的目标是通过成对配准\cite{FGR}\cite{DGR}从多个片段生成全局一致的重建。多实例配准需要不仅从噪声对应关系中剔除异常值,还要识别出各个实例的内点集,使其比经典配准问题更具挑战性。

\textbf{3D目标检测和实例分割}与多实例3D配准密切相关。给定一个单独的点云,3D目标检测\cite{VoteNet}的目的是获取每个感兴趣对象的边界框,而3D实例分割\cite{SGPN, Occuseg}为每个点生成实例标签。尽管它们生成的结果\cite{avetisyan2019end, Scenecad}与多实例配准类似,但它们需要将特定对象或类别的先验训练到网络中。相比之下,多实例配准通过直接将源点云与目标点云中的多个实例对齐来处理两个点云,而不使用关于输入3D扫描内容的任何先验知识。

\textbf{多模型拟合}
多实例配准可以通过多模型拟合方法来解决,其目的是从多模型生成的数据点中估计模型参数。
% 例如,给定一组从不同线条或圆圈采样的2D点,多模型拟合的目的是估计每条线或圆圈的参数。
现有的多模型拟合方法可以分为基于聚类的方法和基于RANSAC的方法。基于聚类的方法(例如\cite{Tlinkage, RPA, Coverage})通过采样点初始化一个庞大的假设集,然后为每个点计算关于这些假设的偏好向量。根据它们的偏好向量对这些数据点进行聚类。最后,从不同的簇计算模型参数。基于RANSAC的方法(如\cite{PEARL, MultiX, ProgressiveX, MCT, CONSAC, ProgressiveX2})顺序运行修订后的RANSAC以获得多个模型参数。它们在每次迭代中改变每个点的采样权重以获得不同的模型参数。CONSAC\cite{CONSAC}是一种基于学习的方法,通过学习为每个点分配采样权重。无论是基于聚类的方法还是基于RANSAC的方法,都依赖于采样有效假设。当模型数量或异常值比例增加时,需要采样很多假设,这使得这些算法效率非常低。

\textbf{3D空间一致性}是3D配准中异常值剔除的重要属性,它在每对点之间通过刚性变换来定义。谱匹配\cite{leordeanu2005spectral}通过在每对对应关系之间使用长度一致性来构建一个图,并从图中提取最大团以剔除异常值。现有的方法如TEASER\cite{TEASER},GORE\cite{bustos2017guaranteed}和PMC\cite{parra2019practical}也将空间一致性纳入到它们的算法中。最近,ROBIN\cite{shi2021robin}将空间一致性的概念推广到更高阶。PointDSC\cite{PointDSC}将空间一致性集成到端到端学习管道中,以便更好地回归内点概率。

受这些工作的启发,我们也采用了空间一致性作为解决方案。与现有方法(如谱聚类\cite{leordeanu2005spectral}或近似解\cite{shi2021robin})在空间一致性图中应用的方法不同,这些方法速度较慢且难以处理多个实例,我们采用了一种高效的算法来在对应关系中找到多个实例。具体来说,我们将距离不变矩阵的行向量或列向量作为对应关系的“特征向量”,并运行自底向上的聚类来从不同实例中获得内点对应关系。我们的方法避免了假设采样,这是现有多模型拟合方法的关键弱点。它还不依赖于任何特定特征来获得点对应关系,因此,如果采用更好的特征(无论是3D特征还是图像特征),性能可以进一步提高。






