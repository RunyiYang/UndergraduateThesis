%%
% The BIThesis Template for Bachelor Paper Translation
%
% 北京理工大学毕业设计(论文) —— 使用 XeLaTeX 编译
%
% Copyright 2020-2023 BITNP
%
% This work may be distributed and/or modified under the
% conditions of the LaTeX Project Public License, either version 1.3
% of this license or (at your option) any later version.
% The latest version of this license is in
%   http://www.latex-project.org/lppl.txt
% and version 1.3 or later is part of all distributions of LaTeX
% version 2005/12/01 or later.
%
% This work has the LPPL maintenance status `maintained'.
%
% The Current Maintainer of this work is Feng Kaiyu.
%
% Compile with: xelatex -> biber -> xelatex -> xelatex
%%

\begin{bibprint}

% -------------------------------- 示例内容 ------------------------------------- %
% \textcolor{blue}{参考文献书写规范}

% \textcolor{blue}{参考国家标准《信息与文献参考文献著录规则》【GB/T 7714—2015】,参考文献书写规范如下:}

% \textcolor{blue}{\textbf{1. 文献类型和标识代码}}

% \textcolor{blue}{普通图书:M}\qquad\textcolor{blue}{会议录:C}\qquad\textcolor{blue}{汇编:G}\qquad\textcolor{blue}{报纸:N}

% \textcolor{blue}{期刊:J}\qquad\textcolor{blue}{学位论文:D}\qquad\textcolor{blue}{报告:R}\qquad\textcolor{blue}{标准:S}

% \textcolor{blue}{专利:P}\qquad\textcolor{blue}{数据库:DB}\qquad\textcolor{blue}{计算机程序:CP}\qquad\textcolor{blue}{电子公告:EB}

% \textcolor{blue}{档案:A}\qquad\textcolor{blue}{舆图:CM}\qquad\textcolor{blue}{数据集:DS}\qquad\textcolor{blue}{其他:Z}

% \textcolor{blue}{\textbf{2. 不同类别文献书写规范要求}}

% \textcolor{blue}{\textbf{期刊}}

% \noindent\textcolor{blue}{[序号]主要责任者. 文献题名[J]. 刊名, 出版年份, 卷号(期号): 起止页码. }

% \printbibliography [type=article,heading=none] 

% \textcolor{blue}{\textbf{普通图书}}

% \noindent\textcolor{blue}{[序号]主要责任者. 文献题名[M]. 出版地: 出版者, 出版年. 起止页码. }
% \cite{Raymer1992Aircraft}

% \printbibliography [keyword={book},heading=none] 

% \textcolor{blue}{\textbf{会议论文集}}

% \noindent\textcolor{blue}{[序号]析出责任者. 析出题名[A]. 见(英文用In): 主编. 论文集名[C]. (供选择项: 会议名, 会址, 开会年)出版地: 出版者, 出版年. 起止页码. }
% \cite{sunpinyi}

% \printbibliography [type=inproceedings,heading=none] 

% \textcolor{blue}{\textbf{专著中析出的文献}}

% \noindent\textcolor{blue}{[序号]析出责任者. 析出题名[A]. 见(英文用In): 专著责任者. 书名[M]. 出版地: 出版者, 出版年.起止页码. }
% \cite{luoyun}

% \printbibliography [type=inbook,heading=none] 

% \textcolor{blue}{\textbf{学位论文}}

% \noindent\textcolor{blue}{[序号]主要责任者. 文献题名[D]. 保存地: 保存单位, 年份. }
% \cite{zhanghesheng}
% \cite{Sobieski}

% \printbibliography [keyword={thesis},heading=none] 

% \textcolor{blue}{\textbf{报告}}

% \noindent\textcolor{blue}{[序号]主要责任者. 文献题名[R]. 报告地: 报告会主办单位, 年份. }
% \cite{fengxiqiao}
% \cite{Sobieszczanski}

% \printbibliography [keyword={techreport},heading=none] 

% \textcolor{blue}{\textbf{专利文献}}

% \noindent\textcolor{blue}{[序号]专利所有者. 专利题名[P]. 专利国别: 专利号, 发布日期. }
% \cite{jiangxizhou}

% \printbibliography [type=patent,heading=none] 

% \textcolor{blue}{\textbf{国际、国家标准}}

% \noindent\textcolor{blue}{[序号]标准代号. 标准名称[S]. 出版地: 出版者, 出版年. }
% \cite{GB/T16159—1996}

% \printbibliography [keyword={standard},heading=none] 

% \textcolor{blue}{\textbf{报纸文章}}

% \noindent\textcolor{blue}{[序号]主要责任者. 文献题名[N]. 报纸名, 出版年, 月(日): 版次. }
% \cite{xiexide}

% \printbibliography [keyword={newspaper},heading=none] 

% \textcolor{blue}{\textbf{电子文献}}

% \noindent\textcolor{blue}{[序号]主要责任者. 电子文献题名[文献类型/载体类型]. 电子文献的出版或可获得地址(电子文献地址用文字表述), 发表或更新日期/引用日期(任选). }
% \cite{yaoboyuan}

% \printbibliography [keyword={online},heading=none] 

% \textcolor{blue}{关于参考文献的未尽事项可参考国家标准《信息与文献参考文献著录规则》(GB/T 7714—2015)}

% 在使用时,请删除/注释上方示例内容,并启用下方语句以输出所有的参考文献
\printbibliography[heading=none]
\end{bibprint}
