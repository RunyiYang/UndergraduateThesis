%%
% The BIThesis Template for Bachelor Paper Translation
%
% 北京理工大学毕业设计(论文) —— 使用 XeLaTeX 编译
%
% Copyright 2020-2023 BITNP
%
% This work may be distributed and/or modified under the
% conditions of the LaTeX Project Public License, either version 1.3
% of this license or (at your option) any later version.
% The latest version of this license is in
%   http://www.latex-project.org/lppl.txt
% and version 1.3 or later is part of all distributions of LaTeX
% version 2005/12/01 or later.
%
% This work has the LPPL maintenance status `maintained'.
%
% The Current Maintainer of this work is Feng Kaiyu.
%
% Compile with: xelatex -> biber -> xelatex -> xelatex
%%

% 中文摘要
\begin{abstract}
% 中文摘要正文从这里开始
我们研究了如何提取精确的点云配准对应关系的问题。最近的无关键点方法绕过了在低重叠场景中检测可重复关键点的难题,这在配准中显示出巨大的潜力。它们在下采样的超点上寻找对应关系,然后将这些对应关系传播到密集点。这些超点基于其相邻区域是否重叠进行匹配。这种稀疏和松散的匹配需要捕获点云的几何结构的上下文特征。我们提出了几何变换器来学习用于稳健超点匹配的几何特征。它编码了成对距离和三元角度,使其在低重叠情况下稳健,且对刚性变换具有不变性。这种简单的设计令人惊讶地获得了高的匹配精度,以至于在估计对齐变换时不需要 RANSAC,从而加速了$100$倍。我们的方法在具有挑战性的3DLoMatch基准测试中将内点比率提高了$17{\sim}30$个百分点,将配准召回率提高了超过$7$个点。

\end{abstract}
