%%
% The BIThesis Template for Bachelor Paper Translation
%
% 北京理工大学毕业设计(论文) —— 使用 XeLaTeX 编译
%
% Copyright 2020-2023 BITNP
%
% This work may be distributed and/or modified under the
% conditions of the LaTeX Project Public License, either version 1.3
% of this license or (at your option) any later version.
% The latest version of this license is in
%   http://www.latex-project.org/lppl.txt
% and version 1.3 or later is part of all distributions of LaTeX
% version 2005/12/01 or later.
%
% This work has the LPPL maintenance status `maintained'.
%
% The Current Maintainer of this work is Feng Kaiyu.
%
% Compile with: xelatex -> biber -> xelatex -> xelatex
%%

% 第一章节

\chapter{问题陈述}


在多实例点云配准问题中,源点云$\mathbf{X}$提供了一个3D模型的实例,目标点云$\mathbf{Y}$包含了这个模型的$K$个实例,其中这些实例是一组点的集合,这些点可能只采样了3D模型的一部分。如果我们将第$k^{th}$个实例写为$\mathbf{Y}_k$,那么目标点云$\mathbf{Y}$可以分解为$
%\begin{equation}
\mathbf{Y} = \mathbf{Y}_0 \cup \mathbf{Y}_1 \cup \ldots \mathbf{Y}_k \ldots \cup \mathbf{Y}_K$。
%\end{equation}
这里我们使用$\mathbf{Y}_0$表示点云中不属于任何实例的部分。
多实例3D配准的目标是找到刚性变换$(\mathbf{R}_k, \mathbf{t}_k)$,将源实例$\mathbf{X}$对准到每个目标实例$\mathbf{Y}_k$。
如果我们设法获得源实例与每个目标实例$\mathbf{X} \leftrightarrow \mathbf{Y}_k$之间的对应关系,那么通过最小化对齐误差之和(\ref{eq:solve_rigid_transform}) \cite{SVD},可以从对应关系集合$\mathbf{X}\leftrightarrow \mathbf{Y}_k$中求解目标点云中第$k^{th}$个实例的位姿$(\mathbf{R}_k, \mathbf{t}_k)$:
\begin{equation}
\underset{\mathbf{R}_k,\mathbf{t}k}{\min}\sum_i{\parallel}\mathbf{y}{ki}-(\mathbf{R}_k\mathbf{x}_i+\mathbf{t}_k)\parallel ^2.
\label{eq:solve_rigid_transform}
\end{equation}
考虑到我们已经获得了源点云和目标点云之间的一组对应关系$\mathcal{C}$。多实例配准任务的关键是将这些对应关系分类为与不同实例相关的独立集合,即:
\begin{equation}
\mathcal{C} = \mathcal{C}_0 \cup \mathcal{C}_1\cdots \cup \mathcal{C}_K.
\end{equation}
这里,$\mathcal{C}_0$用来表示异常值集合。如我们所见,多实例配准不仅需要剔除异常值对应关系,还需要解决来自不同实例的对应关系的歧义。这个任务并不容易,因为所有实例看起来都一样,而且通常存在大量的异常值对应关系。










