%%
% The BIThesis Template for Bachelor Paper Translation
%
% 北京理工大学毕业设计(论文) —— 使用 XeLaTeX 编译
%
% Copyright 2020-2023 BITNP
%
% This work may be distributed and/or modified under the
% conditions of the LaTeX Project Public License, either version 1.3
% of this license or (at your option) any later version.
% The latest version of this license is in
%   http://www.latex-project.org/lppl.txt
% and version 1.3 or later is part of all distributions of LaTeX
% version 2005/12/01 or later.
%
% This work has the LPPL maintenance status `maintained'.
%
% The Current Maintainer of this work is Feng Kaiyu.
%
% Compile with: xelatex -> biber -> xelatex -> xelatex
%%

% 第一章节

\chapter{引言}
我们研究了用于点云配准的精确对应关系的提取问题。近期的进步主要由基于学习的,基于对应关系的方法主导~\cite{deng2018ppfnet,gojcic2019perfect,choy2019fully,bai2020d3feat,huang2021predator,yu2021cofinet}。这些方法训练神经网络从两个输入的点云中提取点对应关系,然后基于此使用稳健的估计器,例如RANSAC,计算对齐变换。大多数基于对应关系的方法依赖于关键点的检测~\cite{choy2019fully,bai2020d3feat,ao2021spinnet,huang2021predator}。然而,在两个点云中,特别是当它们的重叠区域很小时,检测可重复的关键点是具有挑战性的。这通常导致在推测的对应关系中的内点比率较低。

受到近期图像匹配进展的启发~\cite{rocco2018neighbourhood,zhou2021patch2pix,sun2021loftr},无关键点方法~\cite{yu2021cofinet}将输入的点云下采样成超点,并通过检查它们的局部邻域(区块)是否重叠进行匹配。这种超点(区块)匹配然后传播到单独的点,产生密集的点对应关系。因此,密集点对应关系的准确性高度依赖于超点匹配的准确性。

超点匹配是稀疏和松散的。其优点是它将严格的点匹配降低到松散的区块重叠,从而放宽了可重复性的要求。同时,区块重叠比基于距离的点匹配更可靠和更有信息量,这对于学习对应关系是一个重要的约束。另一方面,超点匹配需要捕获更多的全局上下文。

为此,Transformer~\cite{vaswani2017attention}已被采用~\cite{wang2019deep,yu2021cofinet}以在点云配准中编码上下文信息。然而,普通的Transformer忽视了点云的几何结构,这使得学习的特征在几何上不够区分,并导致大量的异常匹配。尽管我们可以注入位置嵌入~\cite{zhao2021point,yang2019modeling},但基于坐标的编码是变换不变的,这在配准给定任意姿态的点云时是有问题的。

我们主张,为配准任务学习的点Transformer应该使用点云的几何结构,以便提取变换不变的几何特征。我们提出了针对3D点云的Geometric Transformer(简称GeoTransformer),它仅编码点对的距离和点三元组的角度。

对于一个超点,我们通过在所有其他超点之间的距离和角度的基础上几何地“定位”它来学习一个非局部表示。自注意力机制被用来权衡那些锚定超点的重要性。由于距离和角度对刚性变换是不变的,GeoTransformer有效地学习点云的几何结构,即使在低重叠的情况下也能进行高度稳健的超点匹配。

受益于高质量的超点匹配,我们的方法利用最优传输层获得高内点比率的密集点对应关系,以及高度稳健和精确的配准,而无需依赖RANSAC。因此,我们方法的配准部分运行非常快,例如,对于具有5K对应关系的两个点云,只需要0.01秒,比RANSAC快100倍。在室内和室外的基准测试上的大量实验都证明了GeoTransformer的效果。我们的方法在3DLoMatch基准测试上将内点比率提高了17到30个百分点,将配准召回率提高了超过7个点。

我们的主要贡献包括:
\begin{itemize}
  \vspace{-8pt}
  \item 一个快速和精确的点云配准方法,它既不需要关键点,也不需要RANSAC。
  \vspace{-8pt}
  \item 一个几何变换器,它学习了点云的变换不变的几何表示,用于稳健的超点匹配。
  \vspace{-30pt}
  \item 一个重叠感知的圆损失,它根据区块重叠比率重新调整每个超点匹配的损失,以获得更好的收敛性。
\end{itemize}