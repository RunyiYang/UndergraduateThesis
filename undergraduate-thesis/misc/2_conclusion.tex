%%
% The BIThesis Template for Bachelor Graduation Thesis
%
% 北京理工大学毕业设计(论文)结论 —— 使用 XeLaTeX 编译
%
% Copyright 2020-2023 BITNP
%
% This work may be distributed and/or modified under the
% conditions of the LaTeX Project Public License, either version 1.3
% of this license or (at your option) any later version.
% The latest version of this license is in
%   http://www.latex-project.org/lppl.txt
% and version 1.3 or later is part of all distributions of LaTeX
% version 2005/12/01 or later.
%
% This work has the LPPL maintenance status `maintained'.
%
% The Current Maintainer of this work is Feng Kaiyu.
%
% Compile with: xelatex -> biber -> xelatex -> xelatex

\begin{conclusion}
  本文对三维点云配准进行了详尽的调研,从点云配准的数学方法入手,引入了基于深度学习的点云配准和经典的点云处理和配准的模型,并且介绍了多实例点云配准的新任务。

  在本文研究中,本文关注于多实例点云配准的新任务。首先根据距离不变矩阵的列向量编码了大量关于对应关系相关实例的信息。然后通过聚类算法有效地将对应关系聚集到不同的组中,并通过多次迭代改进结果。无论是在仿真数据集还是真实数据集的结果都表明,这个基于聚类的方法在稳健性、准确性和效率方面都明显优于现有的方法。
  
  但是这个解决方案在所讨论的环境下还没有达到完美,所以本文提出了一个新的基于深度学习框架来解决多实例点云配准的问题。本文利用对比学习来学习输入对应关系的分布良好的深度表示。在此基础上,本文在聚类之前使用剪枝策略,以有效地删除异常值对应并将剩余的内部对应关系分配给正确的实例。然后,可以很容易地估计从源点云到每个实例的转换。在仿真数据集和真实世界数据集的广泛实验证实了本文的框架的有效性和其优于现有解决方案的优势。所提出的表示学习和异常值修剪策略有可能应用于成对点云配准。

  本文对基于深度学习的多实例点云配准算法研究取得了一定的结果,但仍有可进一步探索的空间,在以后的研究中,希望能够提出一个全真的大场景数据集,比如城市场景的车辆点云配准任务。并且在多实例点云配准任务中,本文可以尝试更多的聚类算法,比如谱聚类、层次聚类等,来提高多实例点云配准的效果。在以后的研究中,希望能够借助类似 DETR \cite{carion2020end} 的结构来实现完全的端对端算法,能够使用转移矩阵筛选的功能来达到多实例点云配准的效果。
\end{conclusion}

