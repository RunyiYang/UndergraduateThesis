Three-dimensional point cloud registration is a fundamental task in point cloud processing, which plays a crucial role in a wide range of vision-based applications such as robotic arms, autonomous driving, autonomous mobile robots, and Simultaneous Localization and Mapping (SLAM). With the rise of artificial intelligence, metaverse, autonomous driving, and other technologies, the application of deep learning to point cloud registration has become increasingly mature. Current research on point cloud registration mostly focuses on pairwise registration. However, in practical applications, the target scene may contain multiple repeated instances, and we need to estimate multiple rigid transformations between the template point cloud and these repeated instances in the target point cloud, that is, multi-instance point cloud registration.

    This paper conducts in-depth research on the problem of three-dimensional point cloud registration, especially multi-instance point cloud registration. We have surveyed and summarized the domestic and international research status of multi-instance point cloud registration and summarized the advantages and disadvantages of previous studies. Existing solutions require sampling a large number of hypotheses to detect possible instances and reject outliers, and their robustness and efficiency degrade significantly when the number of instances and outliers increases. We propose to directly group the noisy correspondence set into different clusters based on a distance invariance matrix. Through clustering, instances and outliers are automatically identified, achieving robust and fast performance.
    
    Based on this research background, we propose a new multi-instance point cloud registration framework, Efficient Correspondence Clustering for Multi-instance Point Cloud Registration (ECC). After analyzing the point cloud registration of the clustering method, we find that the effect will drop significantly when it has high dependency on the descriptor and a high proportion of outliers. Therefore, we propose Deep Learning-based Multi-instance Point Cloud Registration (DMR), which first uses contrastive learning to learn a good deep representation of the input estimation correspondence. Then, based on these representations, we propose an outlier pruning strategy and a clustering strategy to effectively remove outliers and allocate the remaining correspondences to the correct instances, achieving better results.
    
    In conclusion, our research shows that the deep learning-based multi-instance point cloud registration method has excellent performance in dealing with complex multi-instance point cloud registration problems and provides new research ideas and methods.