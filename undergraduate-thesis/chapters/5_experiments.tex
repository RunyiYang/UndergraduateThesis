\chapter{实验结果与分析}
本章,本文对使用点云对应聚类的方式和基于深度学习的点云配准方法来进行实验。实验的数据集分为仿真数据集和真实数据集。
\section{数据集}
\subsection{仿真数据集}
如第\ref{sec:dataset}节所述,本文从 PointNet++ \cite{qi2017pointnet++} 的预采样 Modelnet40 数据集\cite{sunmodelnet40} 生成一个仿真数据集。
本文将每个点云下采样到 $256$ 个点,并随机生成 $K$ 个变换(在本文的测试中最多为 $20$)来形成一个目标点云。目标点云也与其他物体和随机点混合,以更好地模拟真实世界的情况。

\begin{figure}
        \centering
        \includegraphics[width=1.0\textwidth]{images/DMR_syn.pdf}
        \caption{
                在仿真数据集上的结果。
                在 (a) 和 (b) 中,绿线和红线分别代表内点对应关系和异常值对应关系。
                在 (c) 中,聚类后每个簇中的对应关系以不同的颜色可视化。
                在 (d-f) 中,绿色边界框表示目标点云中实例的真实姿态,红色边界框表示预测。
                目标点云中的变换点云以蓝色可视化。
        }
        \label{fig:DMR}
\end{figure}

\subsection{真实数据集}
Scan2CAD\cite{avetisyan2019scan2cad} 是一个数据集,它将 ShapeNet\cite{chang2015shapenet} CAD 模型与 ScanNet\cite{dai2017scannet} 点云中的对象实例对齐。部分扫描有多个已对齐的 CAD 模型和标注的姿态。本文选择那些包含多个 CAD 模型的扫描作为目标点云,并从 CAD 模型中采样源点云进行测试。本文为配准测试生成了 173 个样本,其中大多数样本包含 $2 \sim 5$ 个实例。
请注意,在每个点云中,Scan2CAD 中仅标注了部分实例。这意味着本文无法使用部分标注的姿态正确评估性能,如准确率和召回率。为解决这个问题,本文仅在目标点云中标注的物体的真实边界框内匹配点以生成对应关系。同样,本文使用 PREDATOR\cite{huang2021predator} 和 D3Feat\cite{bai2020d3feat} 进行点匹配,两者都是使用 Scan2CAD 数据集中的 1028 个训练样本和 187 个验证样本进行微调的。

\begin{figure}[ht]
        \centering  
        \includegraphics[width=\linewidth]{images/MHF1_curve.pdf}
        \caption{(a) 平均 Hit F1 与离群比例关系。 (b) 平均 Hit F1 与实例数量关系(固定离群比例为 $50\%$)。}
        \label{fig:detail-mm}
        %\label{fig:multi-instance}
\end{figure}

\begin{table}[h]
        \centering
        \caption{在Scan2CAD数据集上配准的结果。}
        \resizebox{.75\columnwidth}{!}{
          \begin{tabular}{ccccc} %& $50\%~70\%$ & $70\%~90\%$
          \toprule
          \textbf{Metric}& MHR$\left( \% \right) \uparrow $& MHP$\left( \% \right) \uparrow $& MHF1$\left( \% \right) \uparrow $ & Time$\left( s \right) \downarrow $ \\
          \hline
          & \multicolumn{4}{c}{PREDATOR( estimated outlier ratio : $76.44\%$)} \\
          \hline
          T-Linkage & 2.46 & 3.79 & 2.71 & 1655.0 \\
          Progressive-X & 11.58 & 6.86 & 7.87 & 26.32\\
          CONSAC & 2.66 & 0.35 & 0.62 & 21.35\\
          \textbf{ECC} & \textbf{31.63} & \textbf{29.23} & \textbf{27.04} & \textbf{1.46/0.51} \\
          \hline
                          
          &\multicolumn{4}{c}{ D3Feat ( estimated outlier ratio :  $97.25\%$)} \\
          \hline
          T-Linkage & 0.04 & 0.22 & 0.06 & 2178.43 \\
          Progressive-X & \textbf{0.67} & \textbf{0.30} & \textbf{0.4} & 28.48 \\
          CONSAC & 0 & 0 & 0 & 21.88 \\
          \textbf{ECC} & 0.29 & 0.04 & 0.07 & \textbf{2.13/0.89} \\
        \bottomrule
        \end{tabular}
        }
          
          \label{tab:Scan2CAD-cad}
\end{table}

\section{高效对应聚类的多实例点云配准 ECC}
\begin{figure}[ht]
        \centering  
        \includegraphics[width=0.75\linewidth]{images/ECC_cluster.pdf}
        \caption{ECC方法在两个数据集的聚类可视化结果}
        \label{fig:ECC_cluster}
        %\label{fig:multi-instance}
\end{figure}

\subsection{仿真数据集}
\subsubsection{仿真点对关系}
在这个测试中,本文直接通过混合真实值和离群值来生成输入对应关系。本文测试了不同的离群比例,$10\%\sim50\%$,$50\%\sim70\%$ 和 $70\%\sim90\%$。请注意,对于每个测试样本,离群点是在给定范围内随机抽样的。结果显示在表 \ref{tab:mm} 中。随着离群比例的增加,几乎所有方法的性能都有所下降,但本文的方法下降缓慢,且仍然明显优于其他方法。本文的算法在 CPU 或 GPU 上的速度比现有方法快 10 倍。
本文还在图 \ref{fig:detail-mm}(a) 中绘制了本文的方法在不同离群比例下的 MHF1(Mean Hit F1) 曲线,其中包括 $20$ 个实例。尽管当离群比例非常大时性能迅速下降,但本文的方法在 $70\%$ 的离群比例下仍然能达到 $90.46\%$ 的 MHF1。图 \ref{fig:detail-mm}(b) 显示了在固定离群比例 $50\%$ 的情况下,不同实例数量的 MHF1 曲线。即使存在 $30$ 个实例,本文的方法的 MHF1 也约为 $92.73\%$。

\subsubsection{基于特征的点对关系生成方式}

在此测试中,本文使用 PREDATOR\cite{huang2021predator} 和 D3Feat\cite{bai2020d3feat} 特征匹配来获得点对应关系。这两个特征模型都是在仿真数据上训练的。结果显示在表 \ref{tab:realcorr} 中。值得注意的是,这两个特征都产生了超过 $90\%$ 的高离群点比例。在这样一个具有挑战性的情况下,本文的方法在鲁棒性和效率方面仍然表现良好,且明显优于现有方法。使用 D3Feat 的结果要比使用 PREDATOR 的结果差得多。原因不仅是因为更多的离群点,而且本文在检查结果时发现缺少内点。在后面的实验中本文在图 \ref{fig:ECC_cluster} 中展示了一些结果。

\begin{table}[ht]
        \centering
        \caption{在不同离群比例的仿真对应关系上的结果。$\uparrow$ 表示越大越好,而 $\downarrow$ 表示相反。本文方法在 CPU/GPU 上的运行时间也展示出来。}
        \resizebox{.75\columnwidth}{!}{
        \begin{tabular}{ccccc} %& $50\%~70\%$ & $70\%~90\%$
            \toprule
            % & MHR$\left( \% \right) \uparrow $ & MHRRE $\left( \right) \downarrow $ & MHRTE & Time\\
            \textbf{方法}& MHR$\left( \% \right) \uparrow $ & MHP$\left( \% \right) \uparrow $ & MHF1$\left( \% \right) \uparrow $ & 时间$\left( s \right) \downarrow $\\
            \hline
            \multicolumn{5}{c}{离群点比重 : $10\%\sim50\%$} \\
            \hline
            T-Linkage & 3.05 & 14.80 & 4.65& 57.27 \\
            Progressive-X & 27.91 & 80.28 & 41.04 & 87.25\\
            CONSAC & 0.47 & 0.47 & 0.47 & 9.23  \\
            \textbf{ECC} & \textbf{96.08} & \textbf{99.73} & \textbf{97.03} & \textbf{0.62/0.30} \\ % 前gt_num个预测值的recall
            \hline
            \multicolumn{5}{c}{离群点比重 : $50\%\sim70\%$} \\
            \hline
            %\hline
            % \textbf{metric} & MHR$\left( \% \right) \uparrow $ & RRE$\left( \degree \right) \downarrow $ & RTE$\left( m \right) \downarrow $ & t$\left( s \right) \downarrow $\\
            %\hline
            %\hline
            T-Linkage & 1.33 & 7.00 & 2.05 & 56.90 \\
            Progressive-X & 20.60 & 75.10 & 31.70 & 85.54  \\
            CONSAC & 0.49 & 0.49 & 0.49 & 9.55\\
            \textbf{ECC} & \textbf{93.99} & \textbf{99.49} & \textbf{95.51} & \textbf{0.55/0.28}\\
            \hline
            
            \multicolumn{5}{c}{Outlier ratio : $70\%\sim90\%$} \\
            \hline
            %\hline
            % \textbf{metric} & MHR$\left( \% \right) \uparrow $ & RRE$\left( \degree \right) \downarrow $ & RTE$\left( m \right) \downarrow $ & t$\left( s \right) \downarrow $\\
            %\hline
            %\hline
            T-Linkage & 0.81 & 4.42 & 1.25 & 56.89 \\
            Progressive-X & 12.88 & 62.60 & 20.73 & 84.5 \\
            CONSAC & 0.51 & 0.51 & 0.51 & 7.70  \\
            \textbf{ECC} & \textbf{60.39} & \textbf{94.42} & \textbf{69.36} &\textbf{0.50/0.24}\\
    
            \hline
            \multicolumn{5}{c}{离群点比重 : $90\%\sim99\%$} \\
            \hline
            T-Linkage & 0.28 & 1.30 & 0.42 & 56.69 \\
            Progressive-X & 7.13 & 39.19 & 11.67 & 84.43\\
            CONSAC & 0.51 & 0.51 & 0.51 & 9.57  \\
            \textbf{ECC} & \textbf{14.70} & \textbf{65.20} & \textbf{22.75} & \textbf{0.47/0.21} \\ % 前gt_num个预测值的recall
            \bottomrule
            %outlier ratio & $10\%~50\%$ & $50\%~70\%$ & $70\%~90\%$\\
    
        \end{tabular}
        }
        
        \label{tab:mm}
\end{table}

\begin{table}
        \centering
        \caption{使用特征匹配在仿真数据上生成对应点的结果。部分结果可在图 \ref{fig:ECC_cluster} 中进行可视化。}
        \resizebox{0.75\textwidth}{!}{
            \begin{tabular}{ccccc} 
                \toprule
                \textbf{Metric}& MHR$\left( \% \right) \uparrow $& MHP$\left( \% \right) \uparrow $& MHF1$\left( \% \right) \uparrow $ & Time$\left( s \right) \downarrow $\\
                \hline
                & \multicolumn{4}{c}{PREDATOR ( estimated outlier ratio : $94.32\%$)} \\
                \hline
                T-Linkage & 0.19 & 0.54 & 0.27 & 43.46 \\
                Progressive-X & 15.90 & 31.01 & 18.98 & 86.39 \\
                CONSAC & 0.1 & 0.07 & 0.08 & 7.65 \\
                \textbf{ECC} &\textbf{53.39} & \textbf{61.44} & \textbf{51.80} & \textbf{1.28/0.48} \\
                \hline
                
                &\multicolumn{4}{c}{D3Feat ( estimated outlier ratio : $99.30\%$)} \\
                \hline
                %\hline
                % \textbf{metric} & MHR$\left( \% \right) \uparrow $ & RRE$\left( \degree \right) \downarrow $ & RTE$\left( m \right) \downarrow $ & t$\left( s \right) \downarrow $\\
                %\hline
                %\hline
                T-Linkage & 0.07 & 0.29 & 0.1 & 56.37  \\
                Progressive-X & 4.29 & 15.28 & 5.94 & 87.22 \\
                CONSAC & 0.13 & 0.04 & 0.05 & 9.53 \\
                \textbf{ECC} & \textbf{16.98} & \textbf{27.05} & \textbf{17.91} & \textbf{0.68/0.30} \\
                \bottomrule
        \end{tabular}
        }
        
        % \caption{Results on synthetic data using feature matching to generate correspondences.$\uparrow$ means the larger the better, while $\downarrow$ indicates the contrary. The running time on CPU/GPU of our method is presented. Some results are visualized in Figure \ref{fig:predatormm}.}
        \label{tab:realcorr}
\end{table}

\subsection{真实数据集}
在真实数据集中的测试结果如表 \ref{tab:Scan2CAD-cad} 所示。本文的方法在使用 PREDATOR 时明显优于现有方法。请注意,当使用 D3Feat 时,所有方法的性能都很差。在仔细检查结果后,本文发现原因不仅是高异常值比率(约 $97.25\%$),而且在使用 D3Feat 时,即使特征匹配仅限于目标点云中的真实边界框内,内点也不足。一些结果如图 \ref{fig:DMR_real}(e) 所示。在(a)和(b)中,绿线和红线分别代表内点对应关系和异常值对应关系。
在(c)中,聚类后每个簇中的对应关系以不同的颜色可视化。
在(d-f)中,绿色边界框表示目标点云中实例的真实姿态,红色边界框表示预测。目标点云中的变换点云以蓝色可视化。


\section{基于深度学习的多实例点云配准 DMR}
\subsection{仿真数据集}
本文首先将基于深度学习的方法与仿真数据集上的其他竞争者进行比较,结果如表 \ref{table1} 所示。
由于内点比例极低,基于采样的方法如 T-Linkage、RansaCov 和 CONSAC 的性能不佳。
由于空间一致性的优势,ECC 表现出有效性。
然而,本文的 DMR 在所有评价指标上均大幅度超越了第二好的方法 ECC。

\setlength{\tabcolsep}{8pt}
\begin{table}[ht]
  \centering
  \caption{在仿真数据集上的配准结果}
  \begin{tabular}{ccccc}
    \hline\noalign{\smallskip}
  & MHR(\%)         & MHP(\%)         & MHF1(\%)         & Time(s)       \\
  \noalign{\smallskip}
  \hline
  \noalign{\smallskip}
  T-linkage  & 0.61           & 1.48           & 0.87           & 3.89          \\
  RansaCov   & 0.73           & 5.33           & 1.29           & 0.14          \\
  CONSAC     & 1.00           & 7.45           & 1.77           & 0.61          \\
  ECC         & 82.90          & 92.92          & 87.63          & 3.56          \\
  DMR              & \textbf{92.60} & \textbf{99.69} & \textbf{96.01} & \textbf{0.06} \\
  \hline
  \end{tabular}
  \label{table1}
\end{table}

为了定性评估本文的DMR并将其与其他竞争者进行比较,本文在图 \ref{fig:DMR} 中提供了一组可视化。图 \ref{fig:DMR} 的第一行显示了输入对应关系,以及本文的剪枝和聚类结果。图 \ref{fig:DMR}(b) 显示,本文的DMR惊人地去除了所有的离群点,剩余的对应关系在图 \ref{fig:DMR}(c) 中被很好地聚类。

图 \ref{fig:DMR} 的第二行显示了本文提出的方法和竞争者的配准结果。可以看到,T-Linkage和RansaCov都无法配准任何实例。对于目标点云中的六个实例,CONSAC只成功配准了一个实例。值得注意的是,尽管ECC成功配准了四个实例,但它未能配准右下角的两张桌子。这是因为这两张桌子混在一起,空间一致性不足以区分它们。然而,本文的方法成功配准了所有实例。

\subsection{真实数据集}
\begin{table}[ht]
        \centering
        \caption{在真实数据集上的配准结果}
        \begin{tabular}{ccccc}
        \hline\noalign{\smallskip}
        & MHR(\%)         & MHP(\%)         & MHF1(\%)         & Time(s)       \\
        \noalign{\smallskip}
        \hline
        \noalign{\smallskip}
        T-linkage  & 34.99          & 46.86          & 40.07          & 6.64          \\
        RansaCov   & 60.50          & 33.28          & 42.94          & \textbf{0.07} \\
        CONSAC     & 55.48          & 53.34          & 54.39          & 0.39          \\
        ECC        & 64.66          & 69.73          & 67.10          & 1.84          \\
        DMR   & \textbf{78.10} & \textbf{70.64} & \textbf{74.18} & 0.10        \\ 
        \hline
        \end{tabular}
        \label{tab:DMR_real}
\end{table}
      
如表 \ref{tab:DMR_real} 所示,本文的DMR在所有三个评估指标,MHR, MHP 和MHF 上均优于所有竞争者,速度也很有竞争力。 
ECC和DMR的性能低于仿真数据集,而其他方法的性能高于仿真数据集。 
这是由于实例内点比率分布的变化。 
      
本文还提供了一组可视化,以定性评估本文的DMR并与其他竞争者进行比较。 
图 \ref{fig:DMR_real}的第一行显示了输入对应关系,以及本文的剪枝和聚类结果。 
图 \ref{fig:DMR_real}(b)表明,本文的方法几乎去除了所有的异常值,确保了如图 \ref{fig:DMR_real}(c)所示的后续聚类高效执行。 
对于包含在目标点云中的五个实例,T-Linkage和CONSAC成功配准了两个实例,但T-Linkage的一个预测有大的错误。 
RansaCov成功配准了三个实例。 
由于目标点云中的三把椅子靠得很近,ECC并没有成功配准所有这些实例。 
本文的方法成功完成了所有实例的配准。
      
\begin{figure}
        \centering
        \includegraphics[width=1.0\textwidth]{images/DMR_real.pdf}
        \caption{
          在真实数据集上的结果
        }
        \label{fig:DMR_real}
\end{figure}
      
\subsection{消融实验}
\label{sec:ablation}
\subsubsection{深度表达的消融研究}为了研究本文采用的深度表达的有效性,本文比较了本文的框架在有和无深度表达下的性能。
没有深度表达的版本删除了特征提取器并设置 $S=\beta$,只依赖空间一致性进行剪枝和聚类。
比较结果如表 \ref{tab:ablation} 所示。
可以看到,使用深度表达后,准确性和速度指标都有所提高。
没有深度表达,剪枝的对应关系少,需要聚类的对应关系多,这增加了本文方法的运行时间。
尽管没有深度表达的框架在MHR、MHP和MHF1上的性能较低,但它仍然是一个有竞争力的基线,适合没有训练数据的情况。
这表明了本文提出的剪枝和聚类策略的有效性。

此外,本文在图 \ref{fig:spectral} 中可视化了有无深度表达的聚类结果。
本文选择了一个例子,其目标点云包含三个实例。
本文首先使用一个3维的one-hot向量来表示一个对应关系属于哪个实例。
然后本文使用这些向量来计算相似性矩阵,并使用聚类的结果对这些矩阵进行排列。
可以看到,通过深度表达排列的相似性矩阵比没有深度表达的框架排列的相似性矩阵小得多,因为在前者的情况下,剪枝过程中去除了更多的异常值。
更重要的是,图 \ref{fig:spectral}(b) 中的矩阵显示了三个明显的簇,这些簇对应于三个实例。
相反,图 \ref{fig:spectral}(a) 中的矩阵显示了两个块,其中右下角的块实际上对应于两个实例,如果不使用提出的深度表达,无法成功区分。

\setlength{\tabcolsep}{2pt}
\begin{table}
  \caption{
    深度表达和剪枝的消融研究结果。
    }
  \centering
  \begin{tabular}{cccccc}
    \hline\noalign{\smallskip}
    深度表达 & 剪枝 & MHR(\%)         & MHP(\%)         & MHF1(\%)         & Time(s)       \\
  \noalign{\smallskip}
  \hline
  \noalign{\smallskip}
  & $\checkmark$       & 76.61          & 65.05          & 70.36          & 0.17          \\
  $\checkmark$                   &         & 62.23          & 32.77          & 42.93          & 1.24          \\
  $\checkmark$                &
 $\checkmark$     & \textbf{78.10} & \textbf{70.64} & \textbf{74.18} & \textbf{0.10}\\
  \hline
  \end{tabular}
  \label{tab:ablation}
\end{table}
\setlength{\tabcolsep}{1.4pt}


\begin{figure}
  \centering
  \includegraphics[width=0.8\textwidth]{images/spectral.pdf}
  \caption{
    无深度表达和有深度表达的聚类结果可视化。
    }
    \vspace{-0.5cm}
  \label{fig:spectral}
\end{figure}

\subsubsection{对剪枝的消融研究}
为了定量研究本文的剪枝策略的有效性,本文简单地在本文的方法中去除剪枝策略,然后比较去除前后的性能。
表 \ref{tab:ablation} 显示,如果去除剪枝步骤,本文的方法的性能会大幅下降,这是由于噪声二进制相似性矩阵不符合 \cite{li2007noise} 定义的理想矩阵,光谱聚类无法正确地对应关系进行分组。
这个结果表明,剪枝对后续的聚类至关重要。

\subsubsection{对 RANSAC 迭代次数的消融研究}
\vspace{-0.3cm}
此外,本文还测试了本文的模型在不同的 RANSAC 迭代次数下的性能,结果如表 \ref{tab:ransac} 所示。
可以看到,只用五次迭代就可以得到相当好的结果,而 50 次和 500 次迭代的性能非常接近。
这些结果表明,剪枝和聚类步骤极大地提高了每个实例的内点比例,因此,只使用少量的迭代次数就可以估计出可靠的结果。


\begin{table}
  \caption{
        RANSAC 迭代次数的影响
  }
  \centering
  \begin{tabular}{cccc}
    \hline\noalign{\smallskip}
    RANSAC 次数 & MHR(\%)         & MHP(\%)         & MHF1(\%)          \\
  \noalign{\smallskip}
  \hline
  \noalign{\smallskip}
  5   & 77.39          & 67.76          & 72.26              \\
  50  & 78.10          & \textbf{70.64} & 74.18                \\
  500 & \textbf{79.33} & \textbf{70.64} & \textbf{74.73}  \\  
  \hline
  \end{tabular}
  \label{tab:ransac}
\end{table}

\subsubsection{关于最难负样本消融研究}
\vspace{-0.3cm}
为每个小批量找到最困难的负对会产生更大的计算开销,但本文的消融研究显示这是值得的。
本文将其与使用随机选择的负对训练的相同模型进行比较。
使用最困难的负对(如本方法所述)和使用随机选择的负对的结果如表 \ref{tab:hardest} 所示。
所有的准确性指标都通过在对比学习中使用最困难的负对得到了提升。
通过比较表 \ref{tab:hardest} 和表 \ref{tab:ablation},本文可以发现,在对比表示学习中使用随机负对得到的结果优于不使用深度表达,但差距很小。
此外,本文在表 \ref{tab:hardest} 中收集了特征空间中正对和 top-K 最困难的负对的平均余弦相似性。
结果显示,使用最困难的负对训练的表示在特征空间中更好地分离,更具区分性。

\begin{table}
  \caption{
        用最难的负样本进行训练的消融实验结果。
  }
  \centering
  \begin{tabular}{ccccccc}
    \hline\noalign{\smallskip}
    & MHR(\%) & MHP(\%) & MHF1(\%) & Positive(\%) & Top-1(\%) & Top-10(\%) \\
    \noalign{\smallskip}
    \hline
    \noalign{\smallskip}
    随机  & 76.85  & 67.50  & 71.87  & 95.43 & 96.91 & 91.37  \\
    最难 & \textbf{78.10} & \textbf{70.64} & \textbf{74.18} & 83.96 & 61.32 & 49.65 \\  
    \hline
  \end{tabular}
  \label{tab:hardest}
\end{table}


\section{小结}
本章,本文对提出的两种多实例点云配准模型进行了广泛的实验评估,包括定量的实验评估和可视化的实验评估,并且对基于深度学习的方法进行了广泛的消融实验。其中基于深度学习的方法效果超过了单纯的聚类算法,证明了深度学习方法在多实例点云配准中的有效性,学习鲁棒的点对特征能够大大增大配准的效果。